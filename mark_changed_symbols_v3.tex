{\bfseries {pre_post_recovery_missing_stats_v1.Rnw}}\\

Program to generate tables pre- and post- preprocessing of the data on number of rows with missing observations on various variables and the number of rows with gene symbols that begin with "LOC." Additionally, some chromosomes were initiaslly listed as 'Un' for unknown.\\

\documentclass[11pt]{article}
\usepackage{graphicx, subfig}
\usepackage{float}
\pagenumbering{arabic}
\usepackage{enumerate}
\usepackage{Sweave}
\usepackage{booktabs}
\usepackage[table]{xcolor}
\usepackage{framed}
\usepackage{longtable}
\usepackage{tablefootnote}
\usepackage{amsmath}
\usepackage{hyperref}



\begin{document}
\input{mark_changed_symbols_v3-concordance}
 

\hspace{-1.5em}Gene Network\\
COMT expression correlations in Four Brain Areas\\
Fall, 2017\\



\begin{Schunk}
\begin{Soutput}
$`/Volumes/Macintosh_HD_3/genetics/genenetwork2/func_get_match_length.R`
$`/Volumes/Macintosh_HD_3/genetics/genenetwork2/func_get_match_length.R`$value
function (invec, targ) 
{
    len.out <- length(which(invec == targ))
    return(len.out)
}

$`/Volumes/Macintosh_HD_3/genetics/genenetwork2/func_get_match_length.R`$visible
[1] FALSE


$`/Volumes/Macintosh_HD_3/genetics/genenetwork2/func_chrom_names.R`
$`/Volumes/Macintosh_HD_3/genetics/genenetwork2/func_chrom_names.R`$value
function (input.names) 
{
    temp.A <- sub("CHR_HSCHR", "", input.names)
    temp.Init <- grep("^[X-Y]|^[1-9]?|^[1]{1}[0-9]{1}|^[2]{1}[0-2]{1}[^:punct:]", 
        temp.A, value = TRUE)
    temp.Init.pf <- substr(temp.Init, 1, 2)
    temp.Init.pf <- sub("\\_", "", temp.Init.pf)
    temp.Init.pf[which(temp.Init.pf == "CH")] <- NA
    temp.Init.pf[which(temp.Init.pf == "KI")] <- NA
    return(temp.Init.pf)
}

$`/Volumes/Macintosh_HD_3/genetics/genenetwork2/func_chrom_names.R`$visible
[1] FALSE


$`/Volumes/Macintosh_HD_3/genetics/genenetwork2/func_get_start_pos.R`
$`/Volumes/Macintosh_HD_3/genetics/genenetwork2/func_get_start_pos.R`$value
function (input.start, scaleDiv) 
{
    return(input.start/scaleDiv)
}

$`/Volumes/Macintosh_HD_3/genetics/genenetwork2/func_get_start_pos.R`$visible
[1] FALSE


$`/Volumes/Macintosh_HD_3/genetics/genenetwork2/func_myBioCLite.R`
$`/Volumes/Macintosh_HD_3/genetics/genenetwork2/func_myBioCLite.R`$value
function () 
{
    suggestedVersions <- old.packages()
    installedVersions <- installed.packages()[rownames(suggestedVersions), 
        "Version"]
    suggestedVersions <- suggestedVersions[, "ReposVer"]
    needToInstall <- suggestedVersions != installedVersions
    if (sum(needToInstall) > 0) {
        installThese <- names(suggestedVersions)[which(needToInstall)]
        cat("\ncalling biocLite with this list of packages\n", 
            installThese, "\n")
        biocLite(installThese, suppressUpdates = TRUE, suppressAutoUpdate = TRUE)
    }
    else {
        cat("\nThere are no new packages to install\n")
    }
    cat("\ndone\n\n")
}

$`/Volumes/Macintosh_HD_3/genetics/genenetwork2/func_myBioCLite.R`$visible
[1] FALSE


$`/Volumes/Macintosh_HD_3/genetics/genenetwork2/func_genes_are_nums.R`
$`/Volumes/Macintosh_HD_3/genetics/genenetwork2/func_genes_are_nums.R`$value
function (in.data) 
{
    gene.Symbol.table <- as.data.frame(table(in.data$Symbol))
    num.Or.lett <- grepl("[A-Z]", substring(as.character(gene.Symbol.table[, 
        1]), 1, 1))
    genes.As.nums <- length(subset(num.Or.lett, num.Or.lett == 
        FALSE))
    indx.Symb <- which(in.data$Symbol %in% as.character(gene.Symbol.table[c(genes.As.nums), 
        1]))
    return(indx.Symb)
}

$`/Volumes/Macintosh_HD_3/genetics/genenetwork2/func_genes_are_nums.R`$visible
[1] FALSE


$`/Volumes/Macintosh_HD_3/genetics/genenetwork2/func_genes_are_locs.R`
$`/Volumes/Macintosh_HD_3/genetics/genenetwork2/func_genes_are_locs.R`$value
function (in.data, j) 
{
    loc.INDX <- grep("LOC", substring(in.data[j]$Symbol, 1, 3))
    return(loc.INDX)
}

$`/Volumes/Macintosh_HD_3/genetics/genenetwork2/func_genes_are_locs.R`$visible
[1] FALSE


$`/Volumes/Macintosh_HD_3/genetics/genenetwork2/func_genes2chr.R`
$`/Volumes/Macintosh_HD_3/genetics/genenetwork2/func_genes2chr.R`$value
function (in.Data, mapIndx, lim1, lim2) 
{
    doY <- any.Y(in.Data, lim1, lim2)
    topPosgenes.By.chr <- table(in.Data$loc[lim1:lim2])[c(mapIndx[1:doY])]
    topPosgenes.By.chr.percent <- topPosgenes.By.chr/sum(topPosgenes.By.chr)
    genes.By.chr <- table(in.Data$loc)[c(mapIndx[1:24])]
    genes.By.chr.percent <- genes.By.chr[1:doY]/sum(genes.By.chr[1:doY])
    binom.SD <- sqrt(200 * genes.By.chr.percent * (1 - genes.By.chr.percent))
    topPosgenes.By.chr.enrich <- round(topPosgenes.By.chr.percent/genes.By.chr.percent, 
        2)
    topPosgenes.By.chr.pretab <- data.frame(name = map.Chrname.Chrnum$chrnum[1:23], 
        nums = topPosgenes.By.chr, enrich = topPosgenes.By.chr.enrich)
    topPosgenes.By.chr.pretab <- topPosgenes.By.chr.pretab[, 
        c(1, 3, 5)]
    topPosgenes.By.chr.pretab$expect <- round(200 * genes.By.chr.percent, 
        1)
    topPosgenes.By.chr.pretab <- topPosgenes.By.chr.pretab[, 
        c(1, 2, 4, 3)]
    binom.zval <- (topPosgenes.By.chr.pretab$nums.Freq - topPosgenes.By.chr.pretab$expect)/binom.SD[1:23]
    binom.Pval <- round(2 * pnorm(abs(binom.zval), mean = 0, 
        sd = 1, lower.tail = FALSE), 2)
    topPosgenes.By.chr.pretab$pval <- binom.Pval
    colnames(topPosgenes.By.chr.pretab) <- c("Chromosome", "Frequency", 
        "Expected", "Enrichment", "P value")
    return(topPosgenes.By.chr.pretab)
}

$`/Volumes/Macintosh_HD_3/genetics/genenetwork2/func_genes2chr.R`$visible
[1] FALSE


$`/Volumes/Macintosh_HD_3/genetics/genenetwork2/func_anyY.R`
$`/Volumes/Macintosh_HD_3/genetics/genenetwork2/func_anyY.R`$value
function (in.Data, lim1, lim2) 
{
    doY <- c(24)
    if (is.na(match("Y", in.Data$loc[lim1:lim2]))) {
        doY <- c(23)
    }
    else {
    }
    return(doY)
}

$`/Volumes/Macintosh_HD_3/genetics/genenetwork2/func_anyY.R`$visible
[1] FALSE


$`/Volumes/Macintosh_HD_3/genetics/genenetwork2/func_gene_start_end.R`
$`/Volumes/Macintosh_HD_3/genetics/genenetwork2/func_gene_start_end.R`$value
function (in.Data) 
{
    gene.Starts <- getBM(attributes = c("illumina_humanref_8_v3", 
        "hgnc_symbol", "chromosome_name", "start_position", "end_position", 
        "entrezgene"), filters = "illumina_humanref_8_v3", values = as.character(in.Data), 
        mart = ensembl)
    return(gene.Starts)
}

$`/Volumes/Macintosh_HD_3/genetics/genenetwork2/func_gene_start_end.R`$visible
[1] FALSE


$`/Volumes/Macintosh_HD_3/genetics/genenetwork2/func_chisq_genesBychr.R`
$`/Volumes/Macintosh_HD_3/genetics/genenetwork2/func_chisq_genesBychr.R`$value
function (area.Data, input.Data.tab, mapIndx, lim1, lim2) 
{
    doY <- any.Y(area.Data, lim1, lim2)
    genes.By.chr <- table(area.Data$loc)[c(mapIndx[1:24])]
    genes.By.chr.percent <- genes.By.chr[1:doY]/sum(genes.By.chr[1:doY])
    chsq.genes.byChr <- chisq.test(input.Data.tab$Frequency, 
        y = NULL, correct = FALSE, p = genes.By.chr.percent, 
        rescale.p = FALSE, simulate.p.value = FALSE, B = 2000)
    return(chsq.genes.byChr)
}

$`/Volumes/Macintosh_HD_3/genetics/genenetwork2/func_chisq_genesBychr.R`$visible
[1] FALSE


$`/Volumes/Macintosh_HD_3/genetics/genenetwork2/func_mismatch_hgnc.R`
$`/Volumes/Macintosh_HD_3/genetics/genenetwork2/func_mismatch_hgnc.R`$value
function (area.Data, tm.All.Data, entrez.Data) 
{
    Symb.mismatches.temp <- data.frame(matrix(rep(NA, 4 * dim(area.Data)[1]), 
        ncol = 4))
    for (i in 1:dim(area.Data)[1]) {
        if (is.na(area.Data$Symbol[i]) | is.na(tm.All.Data$Symbol[entrez.Data[i]])) {
            Symb.mismatches.temp[i, 1] <- c(i)
            Symb.mismatches.temp[i, 2] <- c("blank")
            Symb.mismatches.temp[i, 3] <- area.Data$Symbol[i]
            Symb.mismatches.temp[i, 4] <- tm.All.Data$Symbol[entrez.Data[i]]
        }
        else if (!is.na(area.Data$Symbol[i]) & !is.na(tm.All.Data$Symbol[entrez.Data[i]])) {
            if (area.Data$Symbol[i] == tm.All.Data$Symbol[entrez.Data[i]]) {
                Symb.mismatches.temp[i, 1] <- c(i)
                Symb.mismatches.temp[i, 2] <- c("match")
                Symb.mismatches.temp[i, 3] <- area.Data$Symbol[i]
                Symb.mismatches.temp[i, 4] <- tm.All.Data$Symbol[entrez.Data[i]]
            }
            else {
                Symb.mismatches.temp[i, 1] <- c(i)
                Symb.mismatches.temp[i, 2] <- c("mismatch")
                Symb.mismatches.temp[i, 3] <- area.Data$Symbol[i]
                Symb.mismatches.temp[i, 4] <- tm.All.Data$Symbol[entrez.Data[i]]
            }
        }
        else {
        }
    }
    return(Symb.mismatches.temp)
}

$`/Volumes/Macintosh_HD_3/genetics/genenetwork2/func_mismatch_hgnc.R`$visible
[1] FALSE


$`/Volumes/Macintosh_HD_3/genetics/genenetwork2/func_get_miss_Stats.R`
$`/Volumes/Macintosh_HD_3/genetics/genenetwork2/func_get_miss_Stats.R`$value
function (x, i, cnames, rnames) 
{
    miss.Temp <- list()
    miss.Temp <- data.frame(miss.Symb = num.NA(x$Symbol), miss.entrez = num.NA(x$ENTREZID), 
        miss.loc = num.NA(x$loc), miss.bp = num.NA(x$bp))
    return(miss.Temp)
}

$`/Volumes/Macintosh_HD_3/genetics/genenetwork2/func_get_miss_Stats.R`$visible
[1] FALSE


$`/Volumes/Macintosh_HD_3/genetics/genenetwork2/func_strip_chr.R`
$`/Volumes/Macintosh_HD_3/genetics/genenetwork2/func_strip_chr.R`$value
function (in.data, j) 
{
    loc.temp.a <- sub("\\::*\\s[0-9]*\\.[0-9]*", "", in.data[j]$Location)
    loc.temp.b <- lapply(loc.temp.a[j], function(x, j) gsub("Chr", 
        "", x), j)
    return(loc.temp.b)
}

$`/Volumes/Macintosh_HD_3/genetics/genenetwork2/func_strip_chr.R`$visible
[1] FALSE


$`/Volumes/Macintosh_HD_3/genetics/genenetwork2/func_strip_bp.R`
$`/Volumes/Macintosh_HD_3/genetics/genenetwork2/func_strip_bp.R`$value
function (x, l) 
{
    temp.bp <- as.numeric(sub(".*\\:", "", x[l]$Location))
    return(temp.bp)
}

$`/Volumes/Macintosh_HD_3/genetics/genenetwork2/func_strip_bp.R`$visible
[1] FALSE


$`/Volumes/Macintosh_HD_3/genetics/genenetwork2/func_miss_chr_bp.R`
$`/Volumes/Macintosh_HD_3/genetics/genenetwork2/func_miss_chr_bp.R`$value
function (in.data, ji) 
{
    un.num <- list()
    no.bp.num <- list()
    un.num <- lapply(major.Area[ji], function(x, ji) length(which(x$loc == 
        "Un")))
    no.bp.num <- lapply(major.Area[ji], function(x, ji) length(which(x$bp == 
        1)))
    miss.Chr.bp <- data.frame(matrix(c(un.num, no.bp.num), ncol = 2))
    colnames(miss.Chr.bp) <- c("Missing Chr", "Missing Start Positions")
    row.names(miss.Chr.bp) <- stand.Col.names
    miss.Chr.bp.tab <- xtable(miss.Chr.bp, caption = "Number of missing Chromosome names or numbers and number of missing gene start positions.")
    return(miss.Chr.bp.tab)
}

$`/Volumes/Macintosh_HD_3/genetics/genenetwork2/func_miss_chr_bp.R`$visible
[1] FALSE


$`/Volumes/Macintosh_HD_3/genetics/genenetwork2/func_miss_symb_descr.R`
$`/Volumes/Macintosh_HD_3/genetics/genenetwork2/func_miss_symb_descr.R`$value
function (in.data, ji) 
{
    missing.Data.symbol <- list()
    missing.Data.gene <- list()
    missing.Data <- data.frame(Missing_Gene_Symbols = rep(NA, 
        4), Missing_Gene_Descriptions = rep(NA, 4))
    missing.Data.symbol <- lapply(in.data[ji], function(x, ji) sum(is.na(x[ji]$Symbol)))
    missing.Data.gene <- lapply(in.data[ji], function(x, ji) sum(is.na(x[ji]$Description)))
    missing.Data[, 1] <- unlist(missing.Data.symbol)
    missing.Data[, 2] <- unlist(missing.Data.gene)
    row.names(missing.Data) <- stand.Col.names
    num.miss.tab <- xtable(missing.Data, caption = c("Missing gene symbols and gene descriptions by brain area."))
    return(num.miss.tab)
}

$`/Volumes/Macintosh_HD_3/genetics/genenetwork2/func_miss_symb_descr.R`$visible
[1] FALSE


$`/Volumes/Macintosh_HD_3/genetics/genenetwork2/func_write_primary_data.R`
$`/Volumes/Macintosh_HD_3/genetics/genenetwork2/func_write_primary_data.R`$value
function (ax, bx, cx, dx) 
{
    success.write <- data.frame(prefront = NA, cbell = NA, tempor = NA, 
        pons = NA)
    success.write$prefront <- is.null(try(write.csv(as.data.frame(ax), 
        "/Volumes/Macintosh_HD_3/genetics/genenetwork2/comt_correlations_20000_prefrontal_augmentedNew.csv.txt", 
        row.names = FALSE)))
    success.write$cbell <- is.null(try(write.csv(as.data.frame(bx), 
        "/Volumes/Macintosh_HD_3/genetics/genenetwork2/comt_correlations_20000_cerebellum_augmentedNew.csv.txt", 
        row.names = FALSE)))
    success.write$tempor <- is.null(try(write.csv(as.data.frame(cx), 
        "/Volumes/Macintosh_HD_3/genetics/genenetwork2/comt_correlations_20000_temporal_augmentedNew.csv.txt", 
        row.names = FALSE)))
    success.write$pons <- is.null(try(write.csv(as.data.frame(dx), 
        "/Volumes/Macintosh_HD_3/genetics/genenetwork2/comt_correlations_20000_pons_augmentedNew.csv.txt", 
        row.names = FALSE)))
    return(success.write)
}

$`/Volumes/Macintosh_HD_3/genetics/genenetwork2/func_write_primary_data.R`$visible
[1] FALSE


$`/Volumes/Macintosh_HD_3/genetics/genenetwork2/func_write_primary_data_one_set.R`
$`/Volumes/Macintosh_HD_3/genetics/genenetwork2/func_write_primary_data_one_set.R`$value
function (ax, namex) 
{
    success.write <- is.null(try(write.csv(as.data.frame(ax), 
        paste("/Volumes/Macintosh_HD_3/genetics/genenetwork2/comt_correlations_20000_", 
            namex, "_augmentedNew.csv.txt", sep = ""), row.names = FALSE)))
    return(success.write)
}

$`/Volumes/Macintosh_HD_3/genetics/genenetwork2/func_write_primary_data_one_set.R`$visible
[1] FALSE


$`/Volumes/Macintosh_HD_3/genetics/genenetwork2/func_test_getdir.R`
$`/Volumes/Macintosh_HD_3/genetics/genenetwork2/func_test_getdir.R`$value
function () 
{
    cmdArgs = commandArgs(trailingOnly = FALSE)
    needle = "--file="
    match = grep(needle, cmdArgs)
    if (length(match) > 0) {
        return(normalizePath(sub(needle, "", cmdArgs[match])))
    }
    else {
        ls_vars = ls(sys.frames()[[1]])
        if ("fileName" %in% ls_vars) {
            return(normalizePath(sys.frames()[[1]]$fileName))
        }
        else {
            if (!is.null(sys.frames()[[1]]$ofile)) {
                return(normalizePath(sys.frames()[[1]]$ofile))
            }
            else {
                return(normalizePath(rstudioapi::getActiveDocumentContext()$path))
            }
        }
    }
}

$`/Volumes/Macintosh_HD_3/genetics/genenetwork2/func_test_getdir.R`$visible
[1] FALSE


$`/Volumes/Macintosh_HD_3/genetics/genenetwork2/func_obj_size_Mb.R`
$`/Volumes/Macintosh_HD_3/genetics/genenetwork2/func_obj_size_Mb.R`$value
function (xv) 
{
    osize <- round((object.size(xv)/1e+06), 3)
    osize <- c(paste(osize, " Mb", sep = ""))
    return(osize)
}

$`/Volumes/Macintosh_HD_3/genetics/genenetwork2/func_obj_size_Mb.R`$visible
[1] FALSE


$`/Volumes/Macintosh_HD_3/genetics/genenetwork2/func_new_objects.R`
$`/Volumes/Macintosh_HD_3/genetics/genenetwork2/func_new_objects.R`$value
function (xin, xcurr) 
{
    new.ones <- setdiff(xin, xcurr)
    return(new.ones)
}

$`/Volumes/Macintosh_HD_3/genetics/genenetwork2/func_new_objects.R`$visible
[1] FALSE


$`/Volumes/Macintosh_HD_3/genetics/genenetwork2/func_tstamp.R`
$`/Volumes/Macintosh_HD_3/genetics/genenetwork2/func_tstamp.R`$value
function (yin) 
if (getOption("recordTimestamps")) {
    wr.success <- try(write(x = paste(deparse(substitute(yin)), 
        as.POSIXct(origin = "1970-01-01", x = as.double(Sys.time())), 
        sep = ","), file = ".Rtimestamps", append = TRUE))
    if (is.null(wr.success)) {
        write.success <- c("timestamp recorded")
    }
    else {
        write.success <- c("timestamp failed")
    }
    return(write.success)
}

$`/Volumes/Macintosh_HD_3/genetics/genenetwork2/func_tstamp.R`$visible
[1] FALSE


$`/Volumes/Macintosh_HD_3/genetics/genenetwork2/func_countNA.R`
$`/Volumes/Macintosh_HD_3/genetics/genenetwork2/func_countNA.R`$value
function (in.vec) 
{
    out.count.NA <- length(which(is.na(in.vec)))
    return(out.count.NA)
}

$`/Volumes/Macintosh_HD_3/genetics/genenetwork2/func_countNA.R`$visible
[1] FALSE


$`/Volumes/Macintosh_HD_3/genetics/genenetwork2/func_where_NA.R`
$`/Volumes/Macintosh_HD_3/genetics/genenetwork2/func_where_NA.R`$value
function (in.vec) 
{
    posNA <- which(is.na(in.vec))
    return(posNA)
}

$`/Volumes/Macintosh_HD_3/genetics/genenetwork2/func_where_NA.R`$visible
[1] FALSE


$`/Volumes/Macintosh_HD_3/genetics/genenetwork2/func_squareAxes.R`
$`/Volumes/Macintosh_HD_3/genetics/genenetwork2/func_squareAxes.R`$value
function (xvar, yvar, title.Text, xlabel.Text, ylabel.Text, corval, 
    subnum, prob, ptcol) 
{
    options(scipen = -3, digits = 1, width = 60)
    dat <- data.frame(x = c(xvar), y = c(yvar))
    if (prob < 0.001) {
        p.out <- c(" < 0.001")
    }
    else {
        p.out <- toString(prob)
    }
    range.Lims.x <- c(mean(xvar, na.rm = TRUE) - 4 * sd(xvar, 
        na.rm = TRUE), mean(xvar, na.rm = TRUE) + 4 * sd(xvar, 
        na.rm = TRUE))
    range.Lims.y <- c(mean(yvar, na.rm = TRUE) - 4 * sd(yvar, 
        na.rm = TRUE), mean(yvar, na.rm = TRUE) + 4 * sd(yvar, 
        na.rm = TRUE))
    sq.Plt <- ggplot(dat, aes(x = xvar, y = yvar)) + geom_point(color = "blue", 
        size = 0.5) + geom_smooth(method = lm, se = FALSE, size = 0.75) + 
        geom_rug() + geom_abline(intercept = mean(yvar, na.rm = TRUE) + 
        (-sign(corval)) * (mean(xvar, na.rm = TRUE)/sd(xvar, 
            na.rm = TRUE)) * sd(yvar, na.rm = TRUE), slope = sign(corval) * 
        (sd(yvar, na.rm = TRUE)/sd(xvar, na.rm = TRUE)), linetype = "dashed", 
        size = 1) + scale_y_continuous(name = ylabel.Text, limits = range.Lims.y) + 
        scale_x_continuous(name = xlabel.Text, limits = range.Lims.x) + 
        ggtitle(title.Text) + theme_classic() + theme(aspect.ratio = 1) + 
        theme(axis.line = element_line(colour = "grey80", size = 1), 
            panel.border = element_rect(colour = "grey80", fill = NA, 
                size = 2.5)) + theme(plot.title = element_text(color = "black", 
        size = 12, hjust = 0.5)) + theme(axis.title = element_text(color = "black", 
        size = 12)) + theme(axis.text.x = element_text(size = 10), 
        axis.text.y = element_text(size = 10)) + geom_vline(xintercept = mean(xvar, 
        na.rm = TRUE), col = "red", size = 1) + geom_hline(yintercept = mean(yvar, 
        na.rm = TRUE), col = "red", size = 1) + annotate("text", 
        x = mean(xvar, na.rm = TRUE) + 1 * sd(xvar, na.rm = TRUE), 
        y = mean(yvar, na.rm = TRUE) + 3.8 * sd(yvar, na.rm = TRUE), 
        label = c(paste("r = ", toString(corval), sep = "")), 
        color = "black", size = 3, hjust = c(0)) + annotate("text", 
        x = mean(xvar, na.rm = TRUE) + 1 * sd(xvar, na.rm = TRUE), 
        y = mean(yvar, na.rm = TRUE) + 3.5 * sd(yvar, na.rm = TRUE), 
        label = c(paste("p = ", p.out, sep = "")), color = "black", 
        size = 3, hjust = c(0)) + annotate("text", x = mean(xvar, 
        na.rm = TRUE) - 2.8 * sd(xvar, na.rm = TRUE), y = mean(yvar, 
        na.rm = TRUE) + 3.8 * sd(yvar, na.rm = TRUE), label = c(paste("n = ", 
        toString(subnum), sep = "")), color = "black", size = 3)
    return(sq.Plt)
}

$`/Volumes/Macintosh_HD_3/genetics/genenetwork2/func_squareAxes.R`$visible
[1] FALSE


$`/Volumes/Macintosh_HD_3/genetics/genenetwork2/func_histChr.R`
$`/Volumes/Macintosh_HD_3/genetics/genenetwork2/func_histChr.R`$value
function (onlyChr, numChr, BPref, gname, nbins, binw) 
{
    histdata <- hist(onlyChr$BP, nbins, plot = FALSE, right = FALSE)
    max.count <- 1.2 * max(histdata$counts)
    histBP <- ggplot(onlyChr, aes(BP)) + geom_histogram(binwidth = binw, 
        color = cbPalette[6], fill = cbPalette[6]) + scale_x_continuous(name = "Base Pair Position (Mb)", 
        limits = c(0, 75)) + scale_y_continuous(name = "Count", 
        limits = c(0, max.count), breaks = pretty_breaks(n = 5)) + 
        ggtitle(paste("Base Pair Positions of Probes on Chromosome ", 
            as.character(numChr), sep = "")) + theme_classic() + 
        theme(aspect.ratio = 0.618) + theme(axis.line = element_line(colour = "black", 
        size = 0.25)) + theme(plot.title = element_text(color = "black", 
        face = "bold", size = 10, hjust = 0)) + theme(axis.title = element_text(color = "black", 
        size = 10, margin = margin(t = 4, r = 4, b = 0, l = 0))) + 
        theme(axis.text.x = element_text(size = 10, margin = margin(t = 4, 
            r = 0, b = 0, l = 0)), axis.text.y = element_text(size = 10, 
            margin = margin(t = 0, r = 2, b = 0, l = 0))) + geom_vline(xintercept = BPref, 
        col = "red", size = 0.5) + annotate("text", x = BPref + 
        3, y = 0.8 * max.count, label = gname, color = "red", 
        size = 3)
    return(histBP)
}

$`/Volumes/Macintosh_HD_3/genetics/genenetwork2/func_histChr.R`$visible
[1] FALSE


$`/Volumes/Macintosh_HD_3/genetics/genenetwork2/func_onlycisChr.R`
$`/Volumes/Macintosh_HD_3/genetics/genenetwork2/func_onlycisChr.R`$value
function (onlyChrcis, numChr, BPref, gname) 
{
    cisBPpos <- ggplot(onlyChrcis, aes(x = dist, y = pval)) + 
        geom_point(color = cbPalette[6], size = 0.75) + scale_x_continuous(name = paste("Base Pair Position Relative to ", 
        gname, " (Mb)", sep = ""), limits = c((min(onlyChrcis$dist) - 
        5), (max(onlyChrcis$dist) + 5)), breaks = pretty_breaks(n = 8)) + 
        scale_y_continuous(name = "-log10(P-value)") + ggtitle(paste("Base Pair Positions of Positively-Correlated Probes on Chromosome ", 
        numChr, " \n Relative to ", gname, " vs. -log10(P-value)", 
        sep = "")) + theme_classic() + theme(aspect.ratio = 0.618) + 
        theme(axis.line = element_line(colour = "black", size = 0.25)) + 
        theme(plot.title = element_text(color = "black", face = "bold", 
            size = 10, hjust = 0)) + theme(axis.title = element_text(color = "black", 
        size = 10, margin = margin(t = 0, r = 4, b = 4, l = 0))) + 
        theme(axis.text.x = element_text(size = 10, margin = margin(t = 4, 
            r = 0, b = 0, l = 0)), axis.text.y = element_text(size = 10, 
            margin = margin(t = 0, r = 2, b = 0, l = 0))) + geom_vline(xintercept = c(0), 
        col = "red", size = 0.5) + annotate("text", x = c(3), 
        y = 0.9 * max(onlyChrcis$pval), label = c(gname), color = "red", 
        size = 3) + annotate("text", x = c(-2), y = -1, label = c("5'"), 
        color = "black", size = 3) + annotate("text", x = c(2), 
        y = -1, label = c("3'"), color = "black", size = 3)
    return(cisBPpos)
}

$`/Volumes/Macintosh_HD_3/genetics/genenetwork2/func_onlycisChr.R`$visible
[1] FALSE


$`/Volumes/Macintosh_HD_3/genetics/genenetwork2/func_hypergeom_plot.R`
$`/Volumes/Macintosh_HD_3/genetics/genenetwork2/func_hypergeom_plot.R`$value
function (prob.50, prob.200, loCut, hiCut, limseq, callingID) 
{
    library(ggplot2, ggthemes)
    library(ggExtra)
    library(psych)
    library(ggpubr)
    library(knitr)
    library(Hmisc)
    library(openxlsx)
    library(xtable)
    library(magrittr)
    library(tables)
    library(stargazer)
    library(plyr)
    library(rlist)
    library(qqman)
    library(manhattanly)
    library(Cairo)
    library(RColorBrewer)
    library(HGNChelper)
    library(tools)
    library(scales)
    library(devtools)
    library(dplyr)
    prob.N.200.plot <- prob.200[c(seq(1, hiCut, 10)), ]
    prob.N.50.plot <- prob.50[c(seq(1, loCut, 5)), ]
    brk <- limseq
    gold.R <- c(1.61803398875)
    my_title <- expression(paste("Chance Probability that k of the 50 or 200 Most Strongly ", 
        italic("COMT-"), "Correlated", sep = ""))
    subt <- c("Probes in a Reference Brain Area Will Match in Another Brain Areas")
    annot1 <- c("Prefrontal~Cortex~italic(cf)~Temporal~Cortex")
    annot2 <- as.character((expression(paste(italic("N"), " = 20,000 probes", 
        sep = ""))))
    annot3 <- as.character((expression(paste(italic("n"), " = 138 subjects", 
        sep = ""))))
    hyp.plot <- ggplot() + geom_point(data = prob.N.50.plot, 
        aes(x = n.match, y = prob, group = 1), size = 0.6, color = "blue", 
        shape = 1) + geom_point(data = prob.N.200.plot, aes(x = n.match, 
        y = prob, group = 1), size = 0.6, color = "red", shape = 6) + 
        scale_y_continuous(breaks = brk, labels = comma(brk, 
            digits = 1)) + scale_x_continuous(name = "Number of Probes in Common (k)", 
        limits = c(0, hiCut), breaks = seq(0, hiCut, 25)) + theme_classic() + 
        ggtitle(my_title) + ylab(expression(log[10](Probability))) + 
        theme(aspect.ratio = 1/gold.R) + theme(plot.title = element_text(color = "black", 
        size = 8.5, hjust = 0.5)) + theme(axis.title = element_text(color = "black", 
        size = 10, vjust = 1)) + theme(axis.text.x = element_text(size = 8), 
        axis.text.y = element_text(size = 8)) + annotate("text", 
        x = 110, y = 10, label = subt, color = "black", size = 3) + 
        annotate("text", x = 150, y = -100, label = annot2, parse = TRUE, 
            color = "black", size = 3) + annotate("text", x = 148, 
        y = -135, label = annot3, parse = TRUE, color = "black", 
        size = 3) + annotate("text", x = 55, y = -150, label = c("Strongest 50"), 
        parse = FALSE, color = "blue", size = 3) + annotate("text", 
        x = 170, y = -450, label = c("Strongest 200"), parse = FALSE, 
        color = "red", size = 3) + annotate("text", x = 27, y = -470, 
        label = c("Every 5th (red) or 10th (blue) point plotted"), 
        parse = FALSE, color = "black", size = 2) + annotate("text", 
        x = 0.95 * hiCut, y = 0, label = c(paste("unique ID: ", 
            callingID, sep = "")), parse = FALSE, color = "gray50", 
        size = 0.75)
    return(hyp.plot)
}

$`/Volumes/Macintosh_HD_3/genetics/genenetwork2/func_hypergeom_plot.R`$visible
[1] FALSE


$`/Volumes/Macintosh_HD_3/genetics/genenetwork2/func_commaSep.R`
$`/Volumes/Macintosh_HD_3/genetics/genenetwork2/func_commaSep.R`$value
function (in.vec) 
{
    out.Sep <- matrix(rep(NA, ((nchar(in.vec[1, 2]) + 1) * dim(in.vec)[1])), 
        ncol = dim(in.vec)[1])
    for (i in 1:dim(in.vec)[1]) {
        out.Sep[2:dim(out.Sep)[1], i] <- unlist(strsplit(in.vec[[2]][i], 
            ""))
        out.Sep[1, i] <- in.vec[[1]][i]
    }
    return(out.Sep)
}

$`/Volumes/Macintosh_HD_3/genetics/genenetwork2/func_commaSep.R`$visible
[1] FALSE


$`/Volumes/Macintosh_HD_3/genetics/genenetwork2/func_nums2genes.R`
$`/Volumes/Macintosh_HD_3/genetics/genenetwork2/func_nums2genes.R`$value
function (in.data, indx.in) 
{
    new.gene.Symbols <- getBM(attributes = c("hgnc_symbol", "entrezgene"), 
        filters = "entrezgene", values = as.character(in.data$ENTREZID[c(indx.in)]), 
        mart = ensembl)
    map.B.to.A <- match(as.character(in.data$ENTREZID[c(indx.in)]), 
        new.gene.Symbols$entrezgene)
    in.data$Symbol[c(indx.in)] <- new.gene.Symbols$hgnc_symbol[c(map.B.to.A)]
    return(in.data$Symbol[c(indx.in)])
}

$`/Volumes/Macintosh_HD_3/genetics/genenetwork2/func_nums2genes.R`$visible
[1] FALSE


$`/Volumes/Macintosh_HD_3/genetics/genenetwork2/func_raincloudPlots.R`
$`/Volumes/Macintosh_HD_3/genetics/genenetwork2/func_raincloudPlots.R`$value
function (in.data) 
{
    source("geom_flat_violin.R")
    raincloud_theme = theme(text = element_text(size = 10), axis.title.x = element_text(size = 11), 
        axis.title.y = element_text(size = 11), axis.text = element_text(size = 9), 
        axis.text.x = element_text(angle = 0, vjust = 0), legend.title = element_text(size = 16), 
        legend.text = element_text(size = 16), legend.position = "right", 
        plot.title = element_text(lineheight = 0.8, face = "bold", 
            size = 12, hjust = 0.5), panel.border = element_blank(), 
        panel.grid.minor = element_blank(), panel.grid.major = element_blank(), 
        axis.line.x = element_line(colour = "black", size = 0.5, 
            linetype = "solid"), axis.line.y = element_line(colour = "black", 
            size = 0.5, linetype = "solid"))
    lb <- function(x) mean(x) - 3 * sd(x)
    ub <- function(x) mean(x) + 3 * sd(x)
    colNames.stacked.area <- colnames(in.data)
    colnames(in.data)[which(colNames.stacked.area == "Mean")] <- c("MeanExpression")
    gplot <- ggplot(data = in.data, aes(y = MeanExpression, x = brainArea, 
        fill = brainArea)) + geom_flat_violin(position = position_nudge(x = 0.2, 
        y = 0.3), alpha = 0.8) + geom_point(aes(y = MeanExpression, 
        color = brainArea), position = position_jitter(width = 0.1), 
        size = 0.02, alpha = c(0.8)) + geom_boxplot(width = 0.1, 
        guides = FALSE, outlier.shape = NA, alpha = 0.8, color = c(rep("black", 
            4))) + scale_y_continuous(name = c("Mean Expression")) + 
        scale_x_discrete(name = c("Brain Area")) + ggtitle("Mean Expression Level per Gene by Brain Area") + 
        expand_limits(y = 6) + guides(fill = FALSE) + guides(color = FALSE) + 
        scale_color_brewer(palette = "Spectral") + scale_fill_brewer(palette = "Spectral") + 
        coord_flip() + theme_bw() + raincloud_theme
    return(gplot)
}

$`/Volumes/Macintosh_HD_3/genetics/genenetwork2/func_raincloudPlots.R`$visible
[1] FALSE


$`/Volumes/Macintosh_HD_3/genetics/genenetwork2/func_unique_id_generator.R`
$`/Volumes/Macintosh_HD_3/genetics/genenetwork2/func_unique_id_generator.R`$value
function (xstring) 
{
    return <- digest(xstring)
}

$`/Volumes/Macintosh_HD_3/genetics/genenetwork2/func_unique_id_generator.R`$visible
[1] FALSE


$`/Volumes/Macintosh_HD_3/genetics/genenetwork2/func_csf.R`
$`/Volumes/Macintosh_HD_3/genetics/genenetwork2/func_csf.R`$value
function () 
{
    cmdArgs = commandArgs(trailingOnly = FALSE)
    needle = "--file="
    match = grep(needle, cmdArgs)
    if (length(match) > 0) {
        return(normalizePath(sub(needle, "", cmdArgs[match])))
    }
    else {
        ls_vars = ls(sys.frames()[[1]])
        if ("fileName" %in% ls_vars) {
            return(normalizePath(sys.frames()[[1]]$fileName))
        }
        else {
            if (!is.null(sys.frames()[[1]]$ofile)) {
                return(normalizePath(sys.frames()[[1]]$ofile))
            }
            else {
                return(normalizePath(rstudioapi::getActiveDocumentContext()$path))
            }
        }
    }
}

$`/Volumes/Macintosh_HD_3/genetics/genenetwork2/func_csf.R`$visible
[1] FALSE


$`/Volumes/Macintosh_HD_3/genetics/genenetwork2/func_rec_script.R`
$`/Volumes/Macintosh_HD_3/genetics/genenetwork2/func_rec_script.R`$value
function (Rin, ui) 
{
    fileConn <- file(".Rscriptnames")
    fd <- data.frame(read.csv(fileConn))
    flag.exist <- FALSE
    yt <- c(1)
    while (!flag.exist & yt < (dim(fd)[1] + 1)) {
        if (grepl(ui, fd[yt, 2])) {
            flag.exist <- TRUE
        }
        else {
        }
        yt <- inc(yt, )
    }
    if (!flag.exist) {
        wr.success <- try(write(x = paste(Rin, ui, as.POSIXct(origin = "1970-01-01", 
            x = as.double(Sys.time())), sep = ", "), file = ".Rscriptnames", 
            append = TRUE))
        if (is.null(wr.success)) {
            write.success <- c("script name and id recorded")
        }
        else {
            write.success <- c("script name and id failed")
        }
    }
    else {
        write.success <- c("the program has already been recorded")
    }
    return(write.success)
}

$`/Volumes/Macintosh_HD_3/genetics/genenetwork2/func_rec_script.R`$visible
[1] FALSE


$`/Volumes/Macintosh_HD_3/genetics/genenetwork2/func_inc.R`
$`/Volumes/Macintosh_HD_3/genetics/genenetwork2/func_inc.R`$value
function (xin, bump) 
{
    if (missing(bump)) {
        delta <- opt("delta")
    }
    else {
        delta <- bump
    }
    xout <- xin + delta
    return(xout)
}

$`/Volumes/Macintosh_HD_3/genetics/genenetwork2/func_inc.R`$visible
[1] FALSE


$`/Volumes/Macintosh_HD_3/genetics/genenetwork2/func_write_orig_data_one_set.R`
$`/Volumes/Macintosh_HD_3/genetics/genenetwork2/func_write_orig_data_one_set.R`$value
function (ax, namex) 
{
    success.write <- is.null(try(write.csv(as.data.frame(ax), 
        paste("/Volumes/Macintosh_HD_3/genetics/genenetwork2/comt_", 
            namex, "_correlation_results.csv.txt", sep = ""), 
        row.names = FALSE)))
    return(success.write)
}

$`/Volumes/Macintosh_HD_3/genetics/genenetwork2/func_write_orig_data_one_set.R`$visible
[1] FALSE


$`/Volumes/Macintosh_HD_3/genetics/genenetwork2/`
NULL

$`/Volumes/Macintosh_HD_3/genetics/genenetwork2/`
NULL
\end{Soutput}
\end{Schunk}


\begin{Schunk}
\begin{Sinput}
> opts_chunk$set(include=FALSE,
+                echo=FALSE,
+                message=FALSE,
+                warning=FALSE)
> # Function to catch warnings that a sheet has no data yet and returns NAs 
> readPrime = function(x,y,z) {     
+      tryCatch(data.frame(read.xlsx(x,sheet = y)),
+             warning = function(w) {print(paste("no data ", y));
+             return(z)},
+             error = function(e) {print(paste("error reading data", y));
+             return(z)}
+ )
+ }
\end{Sinput}
\end{Schunk}

